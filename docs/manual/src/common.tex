\begin{titlepage}
  \begin{center}

  {\Huge AXIS\_DATA\_TO\_AXIS\_STRING}

  \vspace{25mm}

  \includegraphics[width=0.90\textwidth,height=\textheight,keepaspectratio]{img/AFRL.png}

  \vspace{25mm}

  \today

  \vspace{15mm}

  {\Large Jay Convertino}

  \end{center}
\end{titlepage}

\tableofcontents

\newpage

\section{Usage}

\subsection{Introduction}

\par
All data is converted to a string with prefixs for each of the 3 input ports.
All strings are terminated with a single byte, and each has a delimiter between
them. All output characters will be feed out from the buffer till it is exhausted.
While the core is outputing data it will not be ready to take in any other data.
There is no down clock cycle time between output and input, outside of the time
to output the total amount in the buffer. Essentially no wasted cycles.

\subsection{Dependencies}

\par
The following are the dependencies of the cores.

\begin{itemize}
  \item fusesoc 2.X
  \item iverilog (simulation)
  \item cocotb (simulation)
\end{itemize}

\input{src/fusesoc/depend_fusesoc_info.tex}

\subsection{In a Project}
\par
Simply use this core between a sink and source AXIS devices. This will convert from input data into an output string one character at a time.
Check the code to see if others will work correctly.

\section{Architecture}
\par
The only module is the axis\_data\_to\_axis\_string module. It is listed below.

\begin{itemize}
  \item \textbf{axis\_data\_to\_axis\_string } Implement an algorithm to convert input data to ASCII string (see core for documentation).
\end{itemize}

\par
The only always process converts the input data to a stream of ASCII strings.
\begin{enumerate}
\item If the counter has data and the desitination device is ready, output data and decrement buffer counter.
\item If the input data is valid and the counter is 0, meaning previous ASCII string is exhausted, take input data and create ASCII string.
  \begin{enumerate}
    \item Insert data prefix into buffer.
    \item Using unrolled for loop encode tdata input into ASCII HEX and insert into buffer (4 bit nibbles 0 to F).
    \item Insert delimiter into buffer.
    \item Insert user prefix into buffer.
    \item Using unrolled for loop encode tuser input into ASCII HEX and insert into buffer (4 bit nibbles 0 to F).
    \item Insert delimiter into buffer.
    \item Insert destination prefix into buffer.
    \item Using unrolled for loop encode tdest input into ASCII HEX and insert into buffer (4 bit nibbles 0 to F).
    \item Insert string termination into buffer.
    \item Counter is set to string length.
  \end{enumerate}
\end{enumerate}

Please see \ref{Module Documentation} for more information.

\section{Building}

\par
The AXIS data to AXIS string core is written in Verilog 2001. They should synthesize in any modern FPGA software. The core comes as a fusesoc packaged core and can be
included in any other core. Be sure to make sure you have meet the dependencies listed in the previous section.

\subsection{fusesoc}
\par
Fusesoc is a system for building FPGA software without relying on the internal project management of the tool. Avoiding vendor lock in to Vivado or Quartus.
These cores, when included in a project, can be easily integrated and targets created based upon the end developer needs. The core by itself is not a part of
a system and should be integrated into a fusesoc based system. Simulations are setup to use fusesoc and are a part of its targets.

\subsection{Source Files}

\subsubsection{fusesoc\_info File List}
\begin{itemize}
\item src
	\begin{itemize}
	\item {'src/axis\_data\_to\_axis\_string.v': {'file\_type': 'verilogSource'}}
	\end{itemize}
\item tb
	\begin{itemize}
	\item {'tb/tb\_axis.v': {'file\_type': 'verilogSource'}}
	\end{itemize}
\end{itemize}


\subsection{Targets}

\subsubsection{fusesoc\_info Targets}
\begin{itemize}
\item default
	\begin{itemize}
	\item[$\space$] Info: Default for IP intergration.
	\end{itemize}
\item sim
	\begin{itemize}
	\item[$\space$] Info: Default simulation with const data.
	\end{itemize}
\item sim\_8bit\_count\_data
	\begin{itemize}
	\item[$\space$] Info: Counter data input.
	\end{itemize}
\item sim\_rand\_ready\_8bit\_count\_data
	\begin{itemize}
	\item[$\space$] Info: Counter data input, and random ready input.
	\end{itemize}
\item sim\_cocotb
	\begin{itemize}
	\item[$\space$] Info: Cocotb unit tests
	\end{itemize}
\end{itemize}


\subsection{Directory Guide}

\par
Below highlights important folders from the root of the directory.

\begin{enumerate}
  \item \textbf{docs} Contains all documentation related to this project.
    \begin{itemize}
      \item \textbf{manual} Contains user manual and github page that are generated from the latex sources.
    \end{itemize}
  \item \textbf{src} Contains source files for the core
  \item \textbf{tb} Contains test bench files for iverilog and cocotb
    \begin{itemize}
      \item \textbf{cocotb} testbench files
    \end{itemize}
\end{enumerate}

\newpage

\section{Simulation}
\par
There are a few different simulations that can be run for this core.

\subsection{iverilog}
\par
iverilog is used for simple test benches for quick verification, visually, of the core.

\subsection{cocotb}
\par
Future simulations will use cocotb. This feature is not yet implemented.

\newpage

\section{Module Documentation} \label{Module Documentation}

\par
There is a single async module for this core.

\begin{itemize}
\item \textbf{axis\_data\_to\_axis\_string} AXIS data to AXIS string, convert input data to a ASCII string.\\
\end{itemize}
The next sections document the module in great detail.

